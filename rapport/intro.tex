\chapter{Introduction}

	L'an dernier, deux étudiants en Master 1 Informatique ont développé un \textit{framework} permettant de développer, à l'aide d'automates, des jeux au tour par tour. Les jeux ainsi implémentés peuvent héberger une partie de manière indépendante, contenant les protocoles de communication entre elle-même et le client.
	\\
	
	Le but principal de ce PJI était de concevoir un composant logiciel permettant d'accéder aux différents jeux implémentés, et de les instancier par ce biais. Ceci permettrait alors un accès simplifié aux jeux développés et une centralisation des ressources nécessaires à son bon fonctionnement.
	\\
	
	Au-delà de l'accès public aux ressources, c'est notamment dans une vision plus 'interne' qu'a été proposé ce projet individuel. En effet, l'aboutissement final de ce composant logiciel serait d'héberger des intelligences artificelles, s'affrontant sur les supports développés via le \textit{framework} de l'an dernier.
	\\